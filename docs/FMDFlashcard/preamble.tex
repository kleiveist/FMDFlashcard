% ===================== Sprache & Encoding =====================
\usepackage[ngerman]{babel}
\usepackage[T1]{fontenc}
\usepackage[utf8]{inputenc} % für pdfLaTeX
\usepackage{csquotes}
\usepackage{lmodern}
% ===================== Seitenlayout =====================
\usepackage{geometry}
\geometry{top=2cm, bottom=2cm, left=2cm, right=2cm}

% 1,5 Zeilenabstand
\usepackage{setspace}
\onehalfspacing

% Blocksatz-Feintypografie
\usepackage{microtype}

% ===================== Schrift (modern) =====================
\usepackage[sfdefault]{FiraSans} % modern, gut lesbar
\usepackage{FiraMono}           % terminal-artig
\renewcommand{\familydefault}{\sfdefault}

% ===================== Überschriften exakt 12 pt =====================
\usepackage{titlesec}
\titleformat{\section}{\bfseries\fontsize{12pt}{14pt}\selectfont}{\thesection}{1em}{}
\titleformat{\subsection}{\bfseries\fontsize{12pt}{14pt}\selectfont}{\thesubsection}{1em}{}
\titleformat{\subsubsection}{\bfseries\fontsize{12pt}{14pt}\selectfont}{\thesubsubsection}{1em}{}

% Max. 3 Ebenen nummerieren & im ToC zeigen
\setcounter{secnumdepth}{3}
\setcounter{tocdepth}{3}

% Absätze: 6 pt Abstand, kein Erstzeileneinzug
\setlength{\parskip}{6pt}
\setlength{\parindent}{0pt}

% Fußnoten = 10 pt
\makeatletter
\renewcommand\footnotesize{\@setfontsize\footnotesize{10pt}{12pt}}
\makeatother

% ===================== Kopf-/Fußzeile =====================
\usepackage{fancyhdr}
\pagestyle{fancy}
\fancyhf{}
\fancyfoot[C]{\thepage}
\renewcommand{\headrulewidth}{0pt}

% ===================== Grafiken, Tabellen =====================
\usepackage{graphicx}
\usepackage{booktabs}
\usepackage{array}
\usepackage{float}
\usepackage{threeparttable}
\usepackage{tabularx}

% Farben für Tabellen-Zeilen
\usepackage[table]{xcolor} % \rowcolor in Tabellen

% Moderne Tabellen-Box (wie Codeblöcke)
\usepackage[most]{tcolorbox}
\usepackage{tabularray}
\UseTblrLibrary{booktabs}
\tcbuselibrary{listings,breakable}
\newtcolorbox{tableblock}[1][]{
  enhanced,
  breakable,
  colback=black!3,
  colframe=black!12,
  boxrule=0.4pt,
  arc=2mm,
  left=6pt,right=6pt,top=6pt,bottom=6pt,
  #1
}
\renewcommand{\arraystretch}{1.15}
\setlength{\tabcolsep}{6pt}
% Tabelllinien (tabularray) etwas heller
\renewcommand{\lTblrDefaultHruleColorTl}{black!25}
\renewcommand{\lTblrDefaultVruleColorTl}{black!25} % falls du mal vertikale Linien nutzt

% ===================== Code-Blöcke (hellgrau + Terminal-Look) =====================
\usepackage{listings}

\lstdefinestyle{terminal}{
  language=bash,
  basicstyle=\ttfamily\small,
  columns=fullflexible,
  breaklines=true,
  keepspaces=true,
  showstringspaces=false,
  tabsize=2,
  morecomment=[l]{\#},
  commentstyle=\color{green!50!black}\itshape
}

% Umgebung: \begin{codeblock}...\end{codeblock}
\newtcblisting{codeblock}[1][]{
  listing only,
  breakable,
  colback=black!3,
  colframe=black!12,
  boxrule=0.4pt,
  arc=2mm,
  left=6pt,right=6pt,top=6pt,bottom=6pt,
  listing options={style=terminal},
  #1
}

% Inline-Code (optional): \codeinline{...}
\usepackage{xparse}
\NewDocumentCommand{\codeinline}{m}{\texttt{#1}}

% (optional) sauberere Captions (auch für \caption*)
\usepackage[font=small,labelfont=bf]{caption}
% ===================== Hyperlinks (spät laden) =====================
\usepackage[hidelinks]{hyperref}
\usepackage{url}

% ===================== Eigene Verzeichnisse / Registerzuordnung =====================
\usepackage{tocloft}
\usepackage{etoolbox}

% --- Registerzuordnung (Nier-Berlin) -----------------------------
\newlistof{regentry}{rgt}{Registerzuordnung (Nier\,-\,Berlin)}
\newcommand{\listofregister}{\listof{regentry}{Registerzuordnung (Nier\,-\,Berlin)}}
\newcounter{regtab}
\NewDocumentCommand{\RegisterCategory}{O{} m}{%
  \begingroup
  \def\entry{#2}%
  \ifstrempty{#1}{%
    \stepcounter{regtab}%
    \addcontentsline{rgt}{regentry}{Tab \theregtab\quad \entry}%
  }{%
    \addcontentsline{rgt}{regentry}{Tab #1\quad \entry}%
  }%
  \endgroup
}

% ===================== Literatur: biblatex-apa =====================
\usepackage[
  style=apa,
  backend=biber,
  sorting=nyt,
  giveninits=true,
  maxcitenames=2,
  maxbibnames=20,
  doi=true,
  url=true,
  isbn=true
]{biblatex}
\DeclareLanguageMapping{ngerman}{ngerman-apa}
\addbibresource{references.bib}


% ===================== Footnotes in Grau =====================
\usepackage{xcolor}

% Nimm hier deine bestehende Graufarbe oder definiere eine:
\definecolor{FMDGray}{HTML}{6B6B6B} % ggf. anpassen

\makeatletter

% Ziffer (Marker) im Fließtext grau
\renewcommand{\@makefnmark}{%
  \hbox{\textcolor{FMDGray}{\@textsuperscript{\normalfont\@thefnmark}}}%
}

% Fußnotentext + Marker im Fußnotenbereich grau (mit Hanging-Indent)
\renewcommand\@makefntext[1]{%
  \parindent 1em%
  \noindent
  \hb@xt@1.8em{\hss\textcolor{FMDGray}{\@makefnmark}}%
  \textcolor{FMDGray}{#1}%
}

% Fußnoten-Trennlinie grau (kurz + dünn, wie "UI-Linien")
\renewcommand{\footnoterule}{%
  \kern -3pt
  {\color{FMDGray}\hrule width .35\linewidth height .4pt}
  \kern 2.6pt
}
\makeatother
