\subsection{Quickstart}
Die folgenden Schritte zeigen den schnellsten Weg, um \projektname lokal zu starten. Das Setup wird über das Control-/Checkup-Tooling standardisiert (z.\,B. \codeinline{doctor}, \codeinline{install}, \codeinline{run}). 

\begin{codeblock}[title=Quickstart (Beispiel)]
# git clone git clone https://github.com/octocat/Hello-World.git
git clone <REPO-URL>
cd <PROJEKT-ORDNER>

# optional: Health-Check / Doctor
 cd /<PFAD-SYSTEM-PROJEKT-ORDNER>/FMDFlashcard/tools
./control.py --doctor



# Dependencies installieren / Build vorbereiten
./control.sh install

# Projekt starten (Dev)
./control.sh run
\end{codeblock}


\textit{Hinweis:} Das Beispiel oben zeigt die Befehle für ein Linux-System; unter macOS sind \texttt{git clone} und \texttt{cd} identisch. Unter Windows funktionieren die Befehle in \enquote{Git Bash} wie unter Linux; in PowerShell/CMD ebenfalls, nur das Auflisten des Verzeichnisses erfolgt typischerweise mit \texttt{dir} (PowerShell unterstützt auch \texttt{ls}).



\subsection{Toolchain und Frameworks}
Tabelle~\ref{tab:toolchain} fasst die eingesetzten Werkzeuge zusammen. Versionen sind als Mindestempfehlung zu verstehen und können projektabhängig angepasst werden (z.\,B. via \codeinline{.tool-versions}, \codeinline{rust-toolchain.toml} oder \codeinline{package.json}).

\begin{table}[H]
\centering
\begin{tblr}{
  colspec = {Q[l,wd=0.28\textwidth] Q[c,wd=0.18\textwidth] Q[l,wd=0.54\textwidth]},
  row{1} = {font=\bfseries, bg=black!6},
  row{even} = {bg=black!2},
  rowsep = 4pt,
  leftsep = 6pt,
  rightsep = 6pt
}
\toprule
Tool/Framework & Version & Zweck im Projekt \\
\midrule
Git & >= 2.x & Repository klonen, Branching, Versionsverwaltung \\
VS Code & aktuell & IDE/Editor; empfohlen für konsistente Formatierung und Debugging \\
Rust (rustup, cargo) & >= 1.7x & Backend/Build (abhängig vom Projektanteil in Rust) \\
Node.js & >= 18 LTS & Frontend/Tooling (Build, Dev-Server, Bundling) \\
Paketmanager (pnpm/yarn/npm) & projektspezifisch & Abhängigkeiten installieren, Scripts ausführen \\
Control-Skript (\codeinline{control.sh}) & repo-intern & Standardisierte Befehle: Check, Install, Build, Run \\
\bottomrule
\end{tblr}
\caption{Toolchain-Übersicht}
\label{tab:toolchain}
\end{table}