\subsection{Voraussetzung: Python}
Python ist eine weit verbreitete, plattformübergreifende Programmiersprache, die häufig für Automatisierung, Systemadministration und Tooling eingesetzt wird. In diesem Projekt wird Python primär als \textbf{administrative Unterstützung} genutzt: Installations- und Setup-Schritte werden über Skripte standardisiert, und das Checkup-/Diagnose-Skript verwendet Python, um Systemzustand, Abhängigkeiten und Toolchain konsistent zu prüfen.\cite{python:official}\footnote{\url{https://www.python.org/}}

\textbf{Warum Python zuerst?}
\begin{itemize}
  \item Installationsskripte und Checks können damit auf \textbf{Windows, Linux und macOS} einheitlich ausgeführt werden.
  \item Python eignet sich für robuste Systemabfragen (z.\,B. Pfade, Versionen, verfügbare Tools) und reduziert manuelle Fehler.
  \item Das Projekt nutzt Python nicht als Laufzeitabhängigkeit der Anwendung selbst, sondern als \textbf{Tooling-Schicht} rund um Setup und Wartung.
\end{itemize}

\textbf{Hinweis zur Vorinstallation:}
Auf vielen Linux-Distributionen ist \codeinline{python3} in typischen Desktop-Installationen bereits vorhanden (z.\,B. Ubuntu, Fedora Workstation, openSUSE Leap).
Das ist jedoch nicht garantiert: Bei Minimal-Images oder sehr schlanken Installationen kann Python fehlen (z.\,B. bei einer reinen Arch-\codeinline{base}-Installation).
Auf macOS wird Python nicht zuverlässig mitgeliefert und sollte daher explizit installiert werden.
Für eine reproduzierbare Umgebung wird in jedem Fall empfohlen, die verwendete Python-Version zu prüfen und zu dokumentieren.

\textbf{Beispiel Installationsbefehle:}

\begin{codeblock}[title=Python installieren (Beispiele)]

#  Arch Linux (Details vollstaendiges Setup: siehe Anhang A)
sudo pacman -S python python-pip

# Fedora / RPM-basiert (DNF)
sudo dnf install -y python3 python3-pip

# Ubuntu/Debian
sudo apt update
sudo apt install python3 python3-pip

# macOS (Homebrew)
brew install python

# Windows (Winget)
winget install -e --id Python.Python.3
\end{codeblock}

\subsubsection{Prüfen der Installation}
Nach der Installation sollte die Python-Version überprüft werden. Je nach System ist Python entweder über \codeinline{python} oder \codeinline{python3} erreichbar.

\begin{codeblock}[title=Python-Version prüfen]
python3 --version
# alternativ (falls passend):
python --version
\end{codeblock}