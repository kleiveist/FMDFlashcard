

\begin{figure}[H]
    \centering
    \includegraphics[width=0.35\textwidth]{FMD/image/logo.png}
    \caption{Projektlogo \projektname\ (logo).\cite{Eigendarstellung}}
    \label{fig:zielsetzung-logo}
\end{figure}

\section{Einleitung}
% Was:
Diese Arbeit dokumentiert die Konzeption und Umsetzung des Projekts \projektname, einer Vault-basierten Lern- und Flashcard-Anwendung. Der Schwerpunkt liegt auf der technischen Projektdokumentation (Architektur, Implementierung, Build-/Run-Prozess, Tests und Betrieb), sodass das System nachvollziehbar reproduziert, bewertet und weiterentwickelt werden kann.

% Warum:
Die Dokumentation dient als zentrale Referenz für Entscheidungen und Vorgehensweisen im Projektverlauf. Sie reduziert Einarbeitungszeit, erleichtert Reviews und schafft eine belastbare Grundlage für Wartung, Erweiterungen und spätere Refactorings.

% Ergebnis:
Als Ergebnis entsteht eine strukturierte Projektbeschreibung mit klaren Anforderungen, einem konsistenten Architekturmodell, einem nachvollziehbaren Entwicklungsprozess sowie konkreten Anleitungen für Setup, Nutzung und Betrieb.

\subsection{Motivation}
Digitale Lerninhalte verteilen sich häufig über Notizen, PDFs, Karteikarten-Apps und verschiedene Geräte. Dadurch entstehen Medienbrüche, redundante Inhalte und ein hoher Pflegeaufwand. Insbesondere beim langfristigen Lernen ist es hilfreich, wenn Wissen strukturiert, versionierbar und wiederverwendbar vorliegt.

Das Projekt adressiert dieses Problem durch eine Vault-basierte Organisation der Inhalte (analog zu wissensbasierten Notizsystemen) und verbindet diese mit einer Flashcard-Logik. Ziel ist eine Lösung, die Inhalte konsistent verwaltet, den Lernfortschritt abbildet und gleichzeitig eine einfache Erweiterbarkeit für spätere Funktionen (z.\,B. Synchronisation, Import/Export, Statistiken) ermöglicht.

\subsection{Ziel und Fragestellung}
Ziel des Projekts ist die Entwicklung eines lauffähigen Prototyps einer Lernanwendung, die Lerninhalte in einer klar definierten Datenstruktur (Vault) verwaltet und daraus Flashcards für wiederholtes Lernen ableitet.

Die leitende Fragestellung lautet:
\enquote{Wie kann eine Vault-basierte Lernanwendung so konzipiert und implementiert werden, dass Inhalte reproduzierbar verwaltet, effizient gelernt und technisch wartbar weiterentwickelt werden können?}

\subsection{Beitrag dieses Papers}
Dieses Dokument liefert die für das Projekt wesentlichen Artefakte und Entscheidungen in strukturierter Form:
\begin{itemize}
  \item eine nachvollziehbare Beschreibung der Anforderungen und Zielkriterien,
  \item eine konzeptionelle Architektur (Datenmodell, Komponenten, Schnittstellen),
  \item eine strukturierte Darstellung der Entwicklungsphasen von den Grundlagen bis zum Prototyp,
  \item konkrete Hinweise zu Setup, Build/Run, Konfiguration und Projektstruktur,
  \item eine Zusammenfassung zentraler Entscheidungen, Risiken sowie offener Punkte.
\end{itemize}
