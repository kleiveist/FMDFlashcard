\section{Diskussion und Ausblick}
% Zweck: Einordnen, Grenzen und Zukunft aufzeigen

\subsection{Limitationen des Ansatzes}
% Was:
% - Vereinfachungen in der Simulation (Physik, Grafik, Komplexität der NPC-Psychologie)
% - Grenzen lokaler Modelle (Kontextfenster, Qualität der Entscheidungen)
% - Technische Grenzen (Hardware, Entwicklungszeit)
% Warum:
% - Ehrlicher Blick auf Schwächen und offene Punkte

\subsection{Zukünftige Arbeiten}
% Was:
% - Ausbau der NPC-Persönlichkeiten und Langzeitgedächtnisse
% - Nutzung größerer oder spezialisierter Modelle
% - Integration realer AI-Karten, falls verfügbar
% - Ausbau des Projekts als Standard-Benchmark für die Industrie
% Warum:
% - Zeigen, dass das Konzept skalierbar und weiterentwickelbar ist

\section{Fazit}
% Was:
% - Kurze Zusammenfassung der Vision: LifeLLM als Life-Sim-LLM-System und Hardware-Benchmark
% - Wichtigste Punkte: Ereignisgraph, NPC-Massen, LLM-gesteuerte Ereignisse, Hardware-Argument
% - Abschließende Aussage, warum dieses Projekt eine sinnvolle Richtung für Gaming + KI + Hardware ist
